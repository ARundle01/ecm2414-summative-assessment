\documentclass[a4paper, 11pt] {article}
\usepackage[margin=2cm]{geometry}
\usepackage{placeins}
\begin{document}
\title{ECM2414 Card Game Coursework}
\author{690037391 \& 680039372}
\date{\today}
\maketitle
	\begin{center}
		Marks Split: 50:50
	\end{center}
\pagebreak
\section*{Development Log}
\FloatBarrier
\begin{table}[]
\begin{tabular}{|l|l|l|l|l|}
\hline
Date        & Time     & Duration & Roles                                   & Signature              \\ \hline
 26/10/2020 & 11:30    & 1hr 40m  & Driver: 680039372, Navigator: 690037391 & 680039372 \& 690037391 \\ \hline
 26/10/2020 & 13:20    & 1hr 40m  & Driver: 690037391, Navigator: 680039372 & 680039372 \& 690037391 \\ \hline
 26/10/2020 & 15:00    &          & Driver: 680039372, Navigator: 690037391 & 680039372 \& 690037391 \\ \hline
\end{tabular}
\end{table}
\pagebreak
\section*{Production Code Design}
\subsection*{Preliminary User Input}
The \texttt{CardGame} class will query the user for the initial parameters to setup the game, such as the number of players and the path to the file specifying the cards in the game. Once the system is provided with the correct information, the next phase will start.
\subsubsection*{Number of player query}
If the user doesn't enter a valid input (a positive integer), the system should explain that the input is invalid and it should ask the user for the correct input.
\subsubsection*{Pack file query}
If the user doesn't enter a real path, the system should explain and ask the user to input a real path.
Should a real file be provided, the system should check each line to see if it is a positive integer, and if not, it should explain to the user that the file they provided contains invalid cards.It should then ask the user to provide another path to a valid deck.

Should all the cards in the file be valid, the system then check to see if the amount of cards found is equal to $8n$, $n$ being the number of players in the game. If it has too many or too little cards, the system should explain that to the user and ask for a new file that has the right amount of cards.
\pagebreak
	\section*{Testing Design}
\end{document}